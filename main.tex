\documentclass{article}
\usepackage[]{natbib}

\author{Mickael Temporão, PhD \\ Director of Data Science, Delphia}
\title{Empirical Research using Big Data in Canadian Political Science}

\begin{document}
\maketitle

The quantity of data generated by contemporary digital systems has been growing exponentially in the past years, creating extraordinarily large and complex datasets also known as Big Data \citep{groves2011three, keller2017evolution, lazer2017data}.
These datasets combined with new methods have provided scholars with the ability to study texts, images, video, and audio and have substantially improved the state-of-the-art in many domains \citep{lecun2015deep}.
The past decade has seen Canada pioneering empirical research in the field of Artificial Intelligence fostering advances in Machine Learning, more specifically, establishing Deep Learning as a standard to discover patterns in massive data sets.
But, can Big Data be leveraged to conduct valid Canadian Empirical Political Science Research?
Canadian Political Science scholars have been slower to adapt to this new \textit{opportunity data}\footnote{See \citet{keller2017evolution} for more details on \textit{opportunity data}.}. This can partly be explained by the fact that the knowledge required to leverage those large datasets is broader than classical Statistics and Political Science.
Yet, the generation and collection of valid and reliable data to test theories is at the heart of empirical research in Canadian Political Science and remains a major challenge for the empirical tradition.
Large datasets alone can't solve social problems but can provide opportunities to explore new phenomena, previously invisible to datasets on a smaller scale \citep{grimmer2015we}.

Empirical research in Political Science often relies on surveys and random samplings.
The continuous growth in popularity of alternative modes of communication makes random sampling increasingly untenable.
Compared to classic survey based data, Big Data is characterized by a much lower information to data ratio.
However, classic survey instruments are becoming deprecated as the quality of the data available decreases, while at the same time the quantity of data grows exponentially.
Big Data can be an alternative, but those data are nearly universally non-random which bring new methodological challenges to the table in order to make generalizable claims.
While descriptive inference is often denigrated in Political Science, we need to properly describe this new breadth of data in order to better understand its biases and allow us to have a chance to think about causal inference.

Political Scientists have started make use of this new breadth of data in their research in order to test theories that were previously impossible due to the lack of data \citep{rheault2019word, temporao2018ideological}.
For Big Data to help empirical research in Political Science, scholars need to acknowledge that the field is evolving into something broader than Political Science, including machine learning frameworks, that borrow from recent advances in Statistics and Computer Science.
This new interdisciplinary research field that is emerging is called \textit{Computational Political Science}.
This field of study combines recent advances in Computer Science, Statistics, Linguistics, Information Technology, and many others, to develop new and scalable methods that facilitate the extraction of valuable insights from Big Data and enable the study social problems in Political Science.

Big Data has the potential foster empirical research in Canadian Political Science on three fronts.
First, Big Data can increases the scope of the data available for research, and the challenges associated withe these datasets can lead to increased methodological standards enabling scholars to leverage large-scale heterogeneous non-probabilistic samples \citep{ruths2014social, shiffrin2016drawing}. Second, Big Data can allow empirical studies at Sub-National levels, which are often understudied due to data scarcity, as showcased in recent work on election forecasting allowing daily forecasts to be made at the Provincial or even at Riding level \citep{temporao2019crowdsourcing}.
Finally, Big Data can also help to study new types political actors for which data was not previously available and allow the exploration of heterogeneity within a nation-state across time or or even varying languages \citep{temporao2018ideological, rheault2019word}.
Canada has often been defined by its diversity \citep{dufresne2018symbolic} making it a very good candidate to leverage the breadth and depth of Big Data and explore the heterogeneity in this socially rich context.

\newpage
\bibliographystyle{apsr}
\bibliography{references}

\end{document}
